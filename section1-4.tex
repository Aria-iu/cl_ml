\section{列表}
目前为止我们看到了两个操作list的函数:append和length。下面是更多关于list的处理函数:
\begin{lstlisting}[frame=shadowbox]
> p => (JOHN Q PUBLIC)

> (first p) => JOHN

> (rest p) => (Q PUBLIC)

> (second p) => Q

> (third p) => PUBLIC

> (fourth p) => NIL

> (length p) => 3
\end{lstlisting}

函数first、second、third和fourth的命名很恰当:first返回列表的第一个元素,second给出第二个元素,以此类推。函数rest并不那么明显;它的名字代表“列表中第一个元素之后的其余部分”。符号nil和形式()完全同义;它们都是空列表的表示。nil也用于表示Lisp中的“false”值。因此,(fourth p)是nil,因为p中没有第四个元素。请注意,列表不一定只由原子组成,也可以包含子列表作为元素:

\begin{lstlisting}[frame=shadowbox]
> (setf x '((1st element) 2 (element 3) ((4)) 5))
((1ST ELEMENT) 2 (ELEMENT 3) ((4)) 5)

> (length x) => 5

> (first x) => (1ST ELEMENT)

> (second x) => 2

> (third x) => (ELEMENT 3)

> (fourth x) => ((4))

> (first (fourth x)) => (4)

> (first (first (fourth x))) => 4

> (fifth x) => 5

> (first x) => (1ST ELEMENT)

> (second (first x)) => ELEMENT
\end{lstlisting}

目前为止,我们知道了如何访问list的部分。下面是如何建立新的list:
\begin{lstlisting}[frame=shadowbox]
> p => (JOHN Q PUBLIC)

> (cons 'Mr p) => (MR JOHN Q PUBLIC)

> (cons (first p) (rest p)) => (JOHN Q PUBLIC)

> (setf town (list 'Anytown 'USA)) => (ANYTOWN USA)

> (list p 'of town 'may 'have 'already 'won!) =>
((JOHN Q PUBLIC) OF (ANYTOWN USA) MAY HAVE ALREADY WON!)

> (append p '(of) town '(may have already won!)) =>
(JOHN Q PUBLIC OF ANYTOWN USA MAY HAVE ALREADY WON!)

> p => (JOHN Q PUBLIC)
\end{lstlisting}

函数cons代表“构造”。它接受一个元素和一个列表作为参数,并构造一个新的列表,其第一个是元素,其余的是原始列表。list接受任意数量的元素作为参数,并返回一个按顺序包含这些元素的新列表。我们已经看到append,它类似于list;它接受任意数量的列表作为参数,并将它们全部附加在一起,形成一个大列表。因此,要附加的参数必须是列表,而要列出的参数可以是列表或原子。值得注意的是,这些函数会创建新的列表;他们不会修改旧的。当我们说(append p q)时,效果是创建一个全新的列表,该列表以p中的相同元素开始,p本身保持不变。

现在,让我们远离列表上的抽象函数,考虑一个简单的问题:给定一个人的名字以列表的形式,我们如何提取姓氏?对于(JOHN Q PUBLIC),我们可以只使用函数third,但对于没有中间名的人来说,这是行不通的。Common Lisp中有一个名为last的函数;也许这会奏效。我们可以尝试:
\begin{lstlisting}[frame=shadowbox]
> (last p) => (PUBLIC)

> (first (last p)) => PUBLIC
\end{lstlisting}
事实证明,last反常地返回最后一个元素的列表,而不是最后一个元件本身。因此,我们需要将first和last组合起来,以找出实际的最后一个要素。我们希望能够保存我们所做的工作,并给它一个适当的描述,比如姓氏。我们可以使用setf保存p的姓氏,但这无助于确定任何其他姓氏。相反,我们想定义一个新函数,该函数计算表示为列表的任何名称的姓氏。下一节正是这样做的。