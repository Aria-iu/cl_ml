\section{变量}
我们已经了解了符号计算的一些基础知识。现在,我们继续讨论编程语言最重要的特征:能够根据其他对象定义新对象,并为这些对象命名以供将来使用。在这里,符号再次发挥着重要作用,它们用于命名变量。变量可以取一个值,该值可以是任何Lisp对象。给变量赋值的一种方法是使用setf:
\begin{lstlisting}[frame=shadowbox]
> (setf p '(John Q Public)) => (JOHN Q PUBLIC)
> p => (JOHN Q PUBLIC)
> (setf x 10) => 10
> (+ x x) => 20
> (+ x (length p)) => 13
\end{lstlisting}
在将值(John Q Public)赋值给名为p的变量后,我们可以引用名为p的值。同样,在为名为x的变量赋值后,我们也可以引用x和p。

符号也用于在Common Lisp中命名函数。每个符号都可以用作变量或函数的名称,或两者兼而有之,尽管很少(也可能令人困惑)同时使用符号名称。例如,append和length是命名函数但没有值作为变量的符号,pi不命名函数,而是一个值为3.1415926535897936(或其附近)的变量。

