\section{定义新的函数}
特殊形式defun代表“define function”。在这里用于定义一个名为last-name的新函数:
\begin{lstlisting}[frame=shadowbox]
(defun last-name (name)
"Select the last name from a name represented as a list."
(first (last name)))
\end{lstlisting}

我们为新函数命名为last-name。它有一个由单个参数组成的参数列表:(name)。这意味着函数接受一个参数,我们称之为name。它还有一个文档字符串,说明函数的作用。这在任何计算中都没有使用,但文档字符串是调试和理解大型系统的关键工具。定义的主体是(first (last name))),这是我们之前用来挑选p的姓氏的。不同的是,在这里我们想挑选任何名字的姓氏,而不仅仅是特定名字p的姓氏。

一般来说,函数定义采用以下形式(其中文档字符串是可选的,所有其他部分都是必需的):

\textsl{\emph{(defun function-name (parameter...)     "documentation string"     function-body...)}}

函数名必须是一个符号,参数通常是符号(稍后会解释一些复杂情况),函数体由一个或多个表达式组成,在调用函数时对其进行求值。最后一个表达式作为函数调用的值返回。

一旦我们定义了last-name,我们就可以像使用任何其他Lisp函数一样使用它:
\begin{lstlisting}[frame=shadowbox]
> (last-name p) => PUBLIC

> (last-name '(Rear Admiral Grace Murray Hopper)) => HOPPER

> (last-name '(Rex Morgan MD)) => MD

> (last-name '(Spot)) => SPOT

> (last-name '(Aristotle)) => ARISTOTLE
\end{lstlisting}

\begin{lstlisting}[frame=shadowbox]
(defun first-name (name)
"Select the first name from a name represented as a list."
(first name))

> p => (JOHN Q PUBLIC)

> (first-name p) => JOHN

> (first-name '(Wilma Flintstone)) => WILMA

> (setf names '((John Q Public) (Malcolm X)
(Admiral Grace Murray Hopper) (Spot)
(Aristotle) (A A Milne) (Z Z Top)
(Sir Larry Olivier) (Miss Scarlet))) =>

((JOHN Q PUBLIC) (MALCOLM X) (ADMIRAL GRACE MURRAY HOPPER)
(SPOT) (ARISTOTLE) (A A MILNE) (Z Z TOP) (SIR LARRY OLIVIER)
(MISS SCARLET))

> (first-name (first names)) => JOHN
\end{lstlisting}


